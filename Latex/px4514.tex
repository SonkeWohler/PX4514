\documentclass[11pt]{article}

\usepackage[top=2.0cm]{geometry}
\usepackage[sort&compress]{natbib}
\usepackage{url}
\usepackage{graphics}
\usepackage{graphicx}
\usepackage{textcomp}
\usepackage{gensymb}
\usepackage[font=scriptsize]{caption}
\usepackage{subcaption}
\usepackage{placeins}
\usepackage{needspace}
\usepackage{wrapfig}
\usepackage{csquotes}
\usepackage{sidecap}
\usepackage{subfloat}
\usepackage{framed}
\usepackage{lipsum}
\usepackage{subfiles}
\usepackage[english]{isodate}
\usepackage[parfill]{parskip}
\usepackage{mathtools}
\usepackage{cleveref}

\bibliographystyle{plain}

\title{   Predicting the Decisions of the Setter in Volleyball \\
  \large PX4514--Mathematical Modelling Report
}
\author{Pawe{\l} Kaniowski and S\"onke W\"ohler}
\date{\today}    

\begin{document}
  
  \subfile{sections/Title}

  \clearpage
  
  \abstract
    \begingroup
      \fontsize{8pt}{10pt}\selectfont
    
        The setter model given specified factors to create a specific scenario is able to predict where the setter will decide to set the ball, it is based on the experience of a real setter at international level. Volleyball revolves around a setter, which is the most influential position in a team. The setter controls almost every play, the tempo of the game and the decisions made by that position are vital. Setter statistics and data are both really important when developing own team and assess the opponents. The model assesses the pass by calculating how far the setter has to move, setters point of contact with the ball and how far and difficult to set it is from that point to every position. The model uses few specialised factors like the pressure under which the setter is at and weight of each position which represents both difficulty and success rate of each position. The model successfully predicts the sets almost 46 precent of the time however due to the nature of the setter being unpredictable by varying his sets to include 2nd most optimum position to set to the model can be argued to predict 78 precent of sets with the accuracy required.
    
    \endgroup
    \hrulefill
    
      \subfile{sections/Introduction}
      \subfile{sections/Model}
      \subfile{sections/Evaluation}
      \subfile{sections/Conclusion}
    
    \clearpage
    \section*{Acknowledgements}
    \nocite{*} % remove before final submission
    \bibliography{px4514-refs}
  


\end{document}  