\documentclass[main.tex]{subfiles}
\begin{document}
  \section{The Model}
    
    The model was implemented in Mathematica 11.3 (see \url{https://www.wolfram.com/mathematica/}), and can be viewed at: \url{https://github.com/SonkeWohler/PX4514}.
    
    \subsection{Fundamental Model Structure}
      
      The playing field within the model is represented as a coordinate system as shown in \Cref{fig:field}. Over these coordinates each position has a number of properties detailed in \Cref{tab:positions}. These positions are used to refer to the players independent of their coordinates, where position 0 refers to the setter independent of their position within the rotation. For example, \(x_{0,current}\) refers to the x coordinate of the current setter position.
      \\\\
      The model uses this coordinate system to calculate distances, as in \Cref{equ:distIdeal} i.e. the distance between the setter's current position and the ideal setter coordinates. Tha latter coordinates (see \Cref{tab:positions} Setter) describe the point from which the setter has the most freedom when setting \cite{idealSetter}. Being further away from this position reduces the setter's ability to choose a good set, and forces them to set to the few positions they can reach. Said difficulty is represented by \Cref{equ:easeSet}, which describes how easily the setter can set to a position of their choice.
      
      \begin{subequations}
       \begin{equation}
        distance_{max} = 8 \sqrt{2}
        \label{equ:distMax}
       \end{equation}
       \begin{equation}
         distance_{setter \to Ideal} = \sqrt{|x_{0,current} - x_{0,ideal}|^2 + 
         |y_{0,current} - y_{0,ideal}|^2}
         \label{equ:distIdeal}
        \end{equation}
        \begin{equation}
          ease_{Setting} = 1 - \frac{distance_{setter \to Ideal}}{distance_{max}}
          \label{equ:easeSet}
        \end{equation}
        \label{equ:dist}
      \end{subequations} 
      
      The model also keeps track of the game score in \Cref{equ:score}, in order to add pressure to the setter's decision. The idea is that, as pressure rises, players tend to experiment less and stick to plays known to work better.
    
      \begin{subequations}
        \begin{equation}
          points_{Factor}=(points_{their} - points_{own})\frac{points_{higher}}{25} 
          \label{equ:points}
        \end{equation}
        \begin{equation}
          pressure= max( point_{Factor}, 0) + 1      
          \label{equ:pressure}
        \end{equation}
        \label{equ:score}
      \end{subequations}
      
      These quantities are used in \Cref{equ:finalPosition} to increase or decrease how relevant the second term is, which is affected by the difficulty of the attack from that position.
      
      \begin{SCfigure}
        \centering
        \includegraphics[width=0.35\linewidth]{figures/playingFieldGridLabelled}
        \caption{The coordinate system laid over the playing field. The net is spun at y=0 in the figure. \\
          This system is primarily used to allow the model to take player positions into account, like the distance that the setter must cover to reach the ball from their position.}
        \label{fig:field}
      \end{SCfigure}
    
      \begin{table}
        \centering
        \caption{Basic constants defined for each position. \\
          Note that Position 0 refers to the setter independent of their position within the rotation. \\
          \textbf{Ideal Contact Coordinates} describe where a player contacts the ball if set to by the setter, or the ideal position for the setter to set from. \\
          \textbf{Ease of Attack} (also bias or \(ease_{attack}\)) is used to tune the model with bias for specific positions. \\
          \textbf{Validity} Describes whether the setter is allowed to set to a position. 
        }
        \footnotesize
        \begin{tabular}{ c | c c c }
          \hline
          Position  & Ideal Contact Coordinates & Ease of Attack & Validity \\ \hline \hline
          Setter (0) &  8,1 & 10\% & setter y=1 \\
          1 & 10,3 & 50\% & setter is front court \\
          2 & 11,1 & 90\% & setter is back court \\
          3 & 7,1 & 70\% &  always \\
          4 & 3,1 & 90\% & always \\
          5 & invalid & invalid & never\\
          6 & 7,3 & 40\% & always \\
          \hline 
        \end{tabular}
        \label{tab:positions}
        \normalsize
      \end{table}
      
    \subsection{Scoring Positions}
      
      Ultimately the model scores each position by how likely the setter is to score to it, based on a number of factors defined in \Cref{equ:finalPosition}.\\
      Firstly, invalid positions are scored as 0 by default, see \Cref{equ:validPosition}. Furthermore, the score for each position is affected by two terms, a setter-term and an attacker-term. The setter term is simply the distance between that player and the setter, see \Cref{equ:setPosition}. The attacker term is currently defined in \Cref{tab:positions}, but scaled by \Cref{equ:easeSet} and \Cref{equ:pressure}. As a result higher pressures or more difficult setting conditions reduce the relevance of the attacker term and focus the setter on easy to set to positions.
      
      \begin{subequations}
        \begin{equation}
        valid_{(p)}=\left\{  
        \begin{array}{lr} 
        1 : \text{position \emph{p} is valid,  see \Cref{tab:positions}} \\
        0 : \text{otherwise}
        \end{array}
        \right.
        \label{equ:validPosition}
        \end{equation}
        \begin{equation}
        setScore_{(p)} = 1 - \frac{\sqrt{|x_{0,current} - x_{p,ideal}|^2 + |y_{0,current} - y_{p,ideal}|^2}}{distance_{max}}
        \label{equ:setPosition}
        \end{equation}
        \begin{equation}
        score_{(p)} =  valid_{(p)} (setScore_{(p)} + pressure * ease_{Setting} * ease_{attack})
        \label{equ:finalPosition}
        \end{equation}
        \label{equ:position}
      \end{subequations}
      
\end{document}

%could mention that we do not model:
%\begin{itemize}
%  \item attack success independent of setting
%  \item mistakes in contacting the ball
%  \item players are always in ideal position when contacting the ball
%  \item opposing team behaviour
%\end{itemize}